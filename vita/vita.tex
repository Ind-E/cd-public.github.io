\documentclass[10pt]{article}
\RequirePackage[T1]{fontenc}

\usepackage{times}
\usepackage{calc}
\usepackage[shortcuts]{extdash}
\reversemarginpar
\usepackage[paper=letterpaper,
            %includefoot, % Uncomment to put page number above margin
            marginparwidth=1.2in,     % Length of section titles
            marginparsep=.05in,       % Space between titles and text
            margin=1in,               % 1 inch margins
            includemp]{geometry}
\setlength{\parindent}{0in}
\usepackage{tabto}    
\newcommand\mytab{\tab \hspace{-1.75cm}}
\usepackage[shortlabels]{enumitem}

% Simpler bibsections for CV sections
% (thanks to natbib for inspiration)
%
% * For lists of references with hanging indents and no numbers:
%
%   \begin{bibsection}
%       \item ...
%   \end{bibsection}
%
% * For numbered lists of references (with hanging indents):
%
%   \begin{bibenum}
%       \item ...
%   \end{bibenum}
%
%   Note that bibenum numbers continuously throughout. To reset the
%   counter, use
%
%   \restartlist{bibenum}
%
%   at the place where you want the numbering to reset.

\makeatletter
\newlength{\bibhang}
\setlength{\bibhang}{1em}
\newlength{\bibsep}
 {\@listi \global\bibsep\itemsep \global\advance\bibsep by\parsep}
\newlist{bibsection}{itemize}{3}
\setlist[bibsection]{label=,leftmargin=\bibhang,%
        itemindent=-\bibhang,
        itemsep=\bibsep,parsep=\z@,partopsep=0pt,
        topsep=0pt}
\newlist{bibenum}{enumerate}{3}
\setlist[bibenum]{label=[\arabic*],resume,leftmargin={\bibhang+\widthof{[999]}},%
        itemindent=-\bibhang,
        itemsep=\bibsep,parsep=\z@,partopsep=0pt,
        topsep=0pt}
\let\oldendbibenum\endbibenum
\def\endbibenum{\oldendbibenum\vspace{-.6\baselineskip}}
\let\oldendbibsection\endbibsection
\def\endbibsection{\oldendbibsection\vspace{-.6\baselineskip}}
\makeatother

%% Reference the last page in the page number
%
% NOTE: comment the +LP line and uncomment the -LP line to have page
%       numbers without the ``of ##'' last page reference)
%
% NOTE: uncomment the \pagestyle{empty} line to get rid of all page
%       numbers (make sure includefoot is commented out above)
%
\usepackage{fancyhdr,lastpage}
\pagestyle{fancy}
%\pagestyle{empty}      % Uncomment this to get rid of page numbers
\fancyhf{}\renewcommand{\headrulewidth}{0pt}
\fancyfootoffset{\marginparsep+\marginparwidth}
\newlength{\footpageshift}
\setlength{\footpageshift}
          {0.5\textwidth+0.5\marginparsep+0.5\marginparwidth-2in}
\lfoot{\hspace{\footpageshift}%
       \parbox{4in}{\, \hfill %
                    \arabic{page} of \protect\pageref*{LastPage} % +LP
%                    \arabic{page}                               % -LP
                    \hfill \,}}

% Finally, give us PDF bookmarks
\usepackage{color,hyperref}
\definecolor{darkblue}{rgb}{0.0,0.0,0.3}
\hypersetup{colorlinks,breaklinks,
            linkcolor=darkblue,urlcolor=darkblue,
            anchorcolor=darkblue,citecolor=darkblue}

%%%%%%%%%%%%%%%%%%%%%%%% End Document Setup %%%%%%%%%%%%%%%%%%%%%%%%%%%%


%%%%%%%%%%%%%%%%%%%%%%%%%%% Helper Commands %%%%%%%%%%%%%%%%%%%%%%%%%%%%

%%% HEADING AT TOP OF CURRICULUM VITAE

% The title (name) with a horizontal rule under it
% (optional argument typesets an object right-justified across from name
%  as well)
%
% Usage: \makeheading{name}
%        OR
%        \makeheading[right_object]{name}
%
% Place at top of document. It should be the first thing.
% If ``right_object'' is provided in the square-braced optional
% argument, it will be right justified on the same line as ``name'' at
% the top of the CV. For example:
%
%       \makeheading[\emph{Curriculum vitae}]{Your Name}
%
% will put an emphasized ``Curriculum vitae'' at the top of the document
% as a title. Likewise, a picture could be included:
%
%   \makeheading[{\includegraphics[height=1.5in]{my_picture}}]{Your Name}
%
% the picture will be flush right across from the name. For this example
% to work, make sure the extra set of curly braces is included. Also
% makes ure that \usepackage{graphicx} is somewhere in the preamble.
\newcommand{\makeheading}[2][]%
        {\hspace*{-\marginparsep minus \marginparwidth}%
         \begin{minipage}[t]{\textwidth+\marginparwidth+\marginparsep}%
             {\large \bfseries #2 \hfill #1}\\[-0.15\baselineskip]%
                 \rule{\columnwidth}{1pt}%
         \end{minipage}}

%%% SECTION HEADINGS

% The section headings. Flush left in small caps down pseudo-margin.
%
% Usage: \section{section name}
\renewcommand{\section}[1]{\pagebreak[3]%
    \vspace{1.3\baselineskip}%
    \phantomsection\addcontentsline{toc}{section}{#1}%
    \noindent\llap{\scshape\smash{\parbox[t]{\marginparwidth}{\hyphenpenalty=10000\raggedright #1}}}%
    \vspace{-\baselineskip}\par}

%%% LISTS

% This macro alters a list by removing some of the space that follows the list
% (is used by lists below)
\newcommand*\fixendlist[1]{%
    \expandafter\let\csname preFixEndListend#1\expandafter\endcsname\csname end#1\endcsname
    \expandafter\def\csname end#1\endcsname{\csname preFixEndListend#1\endcsname\vspace{-0.6\baselineskip}}}

% These macros help ensure that items in outer-type lists do not get
% separated from the next line by a page break
% (they are used by lists below)
\let\originalItem\item
\newcommand*\fixouterlist[1]{%
    \expandafter\let\csname preFixOuterList#1\expandafter\endcsname\csname #1\endcsname
    \expandafter\def\csname #1\endcsname{\let\oldItem\item\def\item{\pagebreak[2]\oldItem}\csname preFixOuterList#1\endcsname}
    \expandafter\let\csname preFixOuterListend#1\expandafter\endcsname\csname end#1\endcsname
    \expandafter\def\csname end#1\endcsname{\let\item\oldItem\csname preFixOuterListend#1\endcsname}}
\newcommand*\fixinnerlist[1]{%
    \expandafter\let\csname preFixInnerList#1\expandafter\endcsname\csname #1\endcsname
    \expandafter\def\csname #1\endcsname{\let\oldItem\item\let\item\originalItem\csname preFixInnerList#1\endcsname}
    \expandafter\let\csname preFixInnerListend#1\expandafter\endcsname\csname end#1\endcsname
    \expandafter\def\csname end#1\endcsname{\csname preFixInnerListend#1\endcsname\let\item\oldItem}}

% An itemize-style list with lots of space between items
%
% Usage:
%   \begin{outerlist}
%       \item ...    % (or \item[] for no bullet)
%   \end{outerlist}
\newlist{outerlist}{itemize}{3}
    \setlist[outerlist]{label=\enskip\textbullet,leftmargin=*}
    \fixendlist{outerlist}
    \fixouterlist{outerlist}

% An environment IDENTICAL to outerlist that has better pre-list spacing
% when used as the first thing in a \section
%
% Usage:
%   \begin{lonelist}
%       \item ...    % (or \item[] for no bullet)
%   \end{lonelist}
\newlist{lonelist}{itemize}{3}
    \setlist[lonelist]{label=\enskip\textbullet,leftmargin=*,partopsep=0pt,topsep=0pt}
    \fixendlist{lonelist}
    \fixouterlist{lonelist}

% An itemize-style list with little space between items
%
% Usage:
%   \begin{innerlist}
%       \item ...    % (or \item[] for no bullet)
%   \end{innerlist}
\newlist{innerlist}{itemize}{3}
    \setlist[innerlist]{label=\enskip\textbullet,leftmargin=*,parsep=0pt,itemsep=0pt,topsep=0pt,partopsep=0pt}
    \fixinnerlist{innerlist}

% An environment IDENTICAL to innerlist that has better pre-list spacing
% when used as the first thing in a \section
%
% Usage:
%   \begin{loneinnerlist}
%       \item ...    % (or \item[] for no bullet)
%   \end{loneinnerlist}
\newlist{loneinnerlist}{itemize}{3}
    \setlist[loneinnerlist]{label=\enskip\textbullet,leftmargin=*,parsep=0pt,itemsep=0pt,topsep=0pt,partopsep=0pt}
    \fixendlist{loneinnerlist}
    \fixinnerlist{loneinnerlist}

%%% EXTRA SPACE

% To add some paragraph space between lines.
% This also tells LaTeX to preferably break a page on one of these gaps
% if there is a needed pagebreak nearby.
\newcommand{\blankline}{\quad\pagebreak[3]}
\newcommand{\halfblankline}{\quad\vspace{-0.5\baselineskip}\pagebreak[3]}

%%% FORMATTING MACROS

% Provides a linked \doi{#1} that links doi:#1 to http://dx.doi.org/#1
\usepackage{doi}
% To change the text before the DOI, adjust this command
%\renewcommand\doitext{doi:}

% Provides a linked \url{#1} that doesn't require escape characters
\usepackage{url}

% You can adjust the style \url{} uses here:
% (options are: same, rm, sf, tt; defaults to tt)
\urlstyle{same}

% For \email{ADDRESS}, links ADDRESS to the url mailto:ADDRESS
% (uncomment to typeset the e\-/mail address in typewriter font;
%  otherwise, will be typeset in the \urlstyle above)
%\DeclareUrlCommand\emaillink{\urlstyle{tt}}
\providecommand*\emaillink[1]{\nolinkurl{#1}}
\providecommand*\email[1]{\href{mailto:#1}{\emaillink{#1}}}

\providecommand\BibTeX{{B\kern-.05em{\sc i\kern-.025em b}\kern-.08em \TeX}}
\providecommand\Matlab{\textsc{Matlab}}

% Custom hyphenation rules for words that LaTeX has trouble with
\hyphenation{bio-mim-ic-ry bio-in-spi-ra-tion re-us-a-ble pro-vid-er Media-Wiki}

%%%%%%%%%%%%%%%%%%%%%%%% End Helper Commands %%%%%%%%%%%%%%%%%%%%%%%%%%%

%%%%%%%%%%%%%%%%%%%%%%%%% Begin CV Document %%%%%%%%%%%%%%%%%%%%%%%%%%%%

\begin{document}
\makeheading{Calvin Deutschbein \href{https://www.mypronouns.org/they-them}{(they/them)}}

\section{Contact Information}

% NOTE: Mind where the & separators and \\ breaks are in the following
%       table. Table is one row made up of three parboxes. The left
%       parbox has address info, the middle parbox has a vertical bar,
%       and the right parbox has phone and electronic contact
%       information.
%
% MACROS: \rcollength is the width of the right column of the table
%             (adjust it to your liking; default is 1.85in).
%         \spacewidth is width of area between left and right boxes.
%
\newlength{\rcollength}\setlength{\rcollength}{2.5in}%
\newlength{\spacewidth}\setlength{\spacewidth}{20pt}
%
\begin{tabular}[t]{@{}p{\textwidth-\rcollength-\spacewidth}@{}p{\spacewidth}@{}p{\rcollength}}%

% Address box
\parbox{\textwidth-\rcollength-\spacewidth}{%
%Assistant Professor\\
Assistant Professor of~\href{https://willamette.edu/computer-data-science/programs/computer-science/index.html}{Computer Science}\\
\href{https://willamette.edu/}{Willamette University}\\
\href{https://willamette.edu/computer-data-science/index.html}{Computing \& Data Science}\\
Ford Hall 206\\
900 State Street\\
Salem, OR 97301}

&
% Uncomment to add a vertical bar in middle of contact information
%{\vrule width 0.5pt}
\parbox[m][5\baselineskip]{\spacewidth}{} &

% Non-snail-mail contact information
\parbox{\rcollength}{%
\textit{Work:}\mytab+1-503-370-6486 \\
\textit{E-mail:}\mytab\email{ckdeutschbein@willamette.edu}\\
\textit{E-mail:}\mytab\email{calvindeu@gmail.com}\\
\textit{Website:}\mytab\href{https://cd-public.github.io/}{cd-public.github.io/}}

\end{tabular}

\section{Research Interests}

\textbf{Mining Secure Behavior of Hardware Designs:} specification mining,
data mining, machine learning, computer security, cybersecurity, hardware security,
secure design, security validation, computer architecture
hardware design languages (HDL), information flow tracking (IFT), logics of specification,
register transfer level (RTL), instruction set architecture (ISA), hyperproperties,
reduced instruction set computers (RISC),
complex instruction set computers (CISC), x86, temporal logics, linear temporal
logic (LTL)

\section{Current Academic Appointments}

\textbf{Assistant Professor},
            \href{https://willamette.edu/}{Willamette University}
            \hfill {August 2021 to present}
\begin{innerlist}

\item[]\href{https://willamette.edu/computer-data-science/programs/computer-science/index.html}{Computer Science}
\item[]\href{https://willamette.edu/computer-data-science/index.html}{Computing \& Data Science Programs}
\end{innerlist}

\halfblankline

\section{Previous Academic Appointments}

\textbf{Adjunct Professor},
            \href{https://www.elon.edu/}{Elon University}
            \hfill {Spring 2020}
\begin{innerlist}
\item[]\href{https://www.elon.edu/u/academics/arts-and-sciences/computer-science/}{Department of Computer Science}

\end{innerlist}

\halfblankline

\textbf{Instructor},
            \href{https://www.unc.edu/}{The University of North Carolina at Chapel Hill}
            \hfill {Summer 2018}
\begin{innerlist}
\item[]\href{https://cs.unc.edu/}{Department of Computer Science}
\end{innerlist}

\halfblankline

\textbf{Research Scholar},
            \href{https://www.src.org/}{Semiconductor Research Corporation}
            \hfill {October 2018 to August 2021}
\begin{innerlist}
    \item[] \href{https://www.src.org/student-center/}{SRC Research Scholars Program}
\end{innerlist}

%    \begin{innerlist}
%        \item Tasks:
%            \begin{innerlist}
%                \item \href{https://www.src.org/library/research-catalog/2845.001/}{Tackling the Corner Cases: Finding Security Vulnerabilities in CPU Designs}
%                \item \href{https://www.src.org/library/research-catalog/2993.001/}{Automatically Generating Information Flow Properties}
%            \end{innerlist}
 %   \end{innerlist}
\halfblankline

\section{Education}

\href{https://www.unc.edu/}{\textbf{The University of North Carolina at Chapel Hill}},
Chapel Hill, NC
\begin{outerlist}

\item[] Ph.D.,
        \href{https://cs.unc.edu/}
             {Computer Science},
             August 2021
        \begin{innerlist}
        \item Thesis:\mytab\hspace{-3cm} \emph{Mining Secure Behavior of Hardware Designs}
        \item Advisor:\mytab\hspace{-3cm}
              \href{https://www.cs.unc.edu/~csturton/}
                   {Cynthia Sturton}
        \item Area:\mytab\hspace{-3cm} Hardware Security
        \end{innerlist}

\item[] M.S.,
        \href{https://cs.unc.edu/}
             {Computer Science},
             August 2017
        \begin{innerlist}
        \item Thesis:\mytab\hspace{-3cm} \emph{Multi-core Cyclic Executives
for Safety-Critical Systems}
        \item Advisor:\mytab\hspace{-3cm}
              \href{https://sites.wustl.edu/baruah/}
                   {Sanjoy Baruah}
        \item Area:\mytab\hspace{-3cm} Real-Time Systems
        \end{innerlist}


\halfblankline

\end{outerlist}

\href{https://www.uchicago.edu/}{\textbf{The University of Chicago}},
Chicago, IL
\begin{outerlist}

\item[] B.S.,
        \href{https://cs.uchicago.edu/}
             {Computer Science}, March 2015
        \begin{innerlist}
        \item Thesis:\mytab\hspace{-3cm} \emph{Performance and Energy Limits of a Processor-integrated FFT Accelerator}
        \item Advisor:\mytab\hspace{-3cm}
              \href{http://people.cs.uchicago.edu/~aachien/lssg/people/andrew-chien/}
                   {Andrew A. Chien}
        \item Area:\mytab\hspace{-3cm} Computer Architecture
        \end{innerlist}
        
\item[] B.A.,
        \href{https://mathematics.uchicago.edu/}
             {Mathematics}, March 2015
        
\end{outerlist}

\section{Professional Service}


\begin{outerlist}


	\item Program Committee. Hardware and Architectural Support for Security and Privacy (HASP 2022), co-located with MICRO 2022.  \url{https://haspworkshop.org/2022/committee.html}

	\item Program Committee.  Sixth Workshop on Attacks and Solutions in Hardware Security (ASHES 2022), co-located with ACM CCS 2022. \url{http://ashesworkshop.org/committees}



\end{outerlist}

\section{Research Tools}



\begin{outerlist}

	\item Aphrodite. Willamette University.  \url{https://github.com/wu-jldeyoung/Aphrodite}

	\item Isadora. HyperFloGen. \url{https://github.com/cd-public/Isadora}
	
	\item Astarte.  HW Security @ UNC.  \url{https://github.com/cd-public/Astarte}
	
	\item Undine.  HW Security @ UNC.  \url{https://github.com/cd-public/Undine}
	
	\item Coppelia. HW Security @ UNC.  \url{https://github.com/rzhang2285/Coppelia}

\end{outerlist}
\section{Teaching Materials}


\begin{outerlist}
	\item chiTCP. The UChicago $\chi$-Projects.  \url{http://chi.cs.uchicago.edu/about.html}
	
\end{outerlist}


\section{Refereed Journal Publications}

\begin{bibenum}

	\item C. Deutschbein, A. Meza, F. Restuccia, R. Kastner, C. Sturton.
		Toward Hardware Security Property Generation at Scale
		In: \emph{IEEE Security \& Privacy},
		April 2022. \\
		\doi{10.1109/MSEC.2022.3155376}

    \item R. Zhang, C. Deutschbein, P. Huang, C. Sturton.
        End-to-End Automated Exploit Generation for Processor Security Validation.
        \emph{IEEE Design \& Test Special Issue: Hardware Security Top Picks}. 2021.\\
        \doi{10.1109/MDAT.2021.3063314}

    \item C. Deutschbein, T. Fleming, A. Burns, S. Baruah.
        Multi-core Cyclic Executives for Safety-Critical Systems.
        \emph{Science of Computer Programming}, March 2019. \\
        \doi{10.1016/j.scico.2018.11.004}
        
\end{bibenum}

\vspace{0.1in}

\section{Refereed Conference Publications}

\begin{bibenum}


	\item C. Deutschbein, A. Meza, F. Restuccia, R. Kastner, C. Sturton.
		Isadora: Automated Information Flow Property Generation for Hardware Designs.
		In: \emph{Proceedings of the 5th Workshop on Attacks and Solutions in Hardware Security (ASHES)},
		November 2021. \\
		\doi{10.1145/3474376.3487286}

    \item C. Deutschbein, C. Sturton.
        Evaluating Security Specification Mining for a CISC Architecture.
        In: \emph{Proceedings of the IEEE International Symposium on Hardware Oriented Security and Trust (HOST)},
            December 2020. \\
            \doi{10.1109/HOST45689.2020.9300291}

    \item R. Zhang, C. Deutschbein, P. Huang, C. Sturton.
        End-to-End Automated Exploit Generation for Processor Security Validation.
        In: \emph{MICRO-51: Proceedings of the 51st Annual IEEE/ACM International Symposium on Microarchitecture},
            October 2018. \\
            \doi{10.1109/MICRO.2018.00071}
            
    \item C. Deutschbein, T. Fleming, A. Burns, S. Baruah.
        Multi-core Cyclic Executives for Safety-Critical Systems.
        In: \emph{Proceedings of the Third International Symposium on Dependable Software Engineering: Theories, Repositorys, and Applications, SETTA 2017},
            October 2017. \\
            \doi{10.1016/j.scico.2018.11.004}

    \item C. Deutschbein, S. Baruah.
        Preemptive Uniprocessor EDF Schedulability Analysis with Preemption Costs Considered.
        In: \emph{Proceedings of the 2016 IEEE Real-Time Systems Symposium (RTSS)},
            November 2016. \\
            \doi{10.1109/RTSS.2016.047}

    \item T. Thanh-Hoang, A. Shambayati, C. Deutschbein, H. Hoffmann, A. A. Chien
        Performance and energy limits of a processor-integrated FFT accelerator.
        In: \emph{Proceedings of the 2014 IEEE High Performance Extreme Computing Conference (HPEC)},
            September 2014. \\
            \doi{10.1109/HPEC.2014.7040951}

\end{bibenum}

\section{Invited Talks}

\begin{bibenum}


	\item “Who ya gonna call?”: Cybersecurity for the Spectre Era.
    Pacific University Mathematics, Computer Science, and Data Science Colloquium.
    17 November, 2022.

	\item Isadora: Automated Information Flow Property Generation for Hardware Designs.
	3rd Annual INTEL Side Channel Academic Program Workshop 2021. 11 November 2021.
	
	\item Isadora: Automated Information Flow Property Generation for Hardware Designs.
	Workshop on Secure RISC-V Architecture Design (secrisc-v'21). 7 November 2021.
	
	\item Creating Information Flow Specifications.
	Radix Presentation for Tortuga Logic. 20 August 2021.

    \item Extracting IF specifications from HW designs. 
    University of Illinois--Urbana Champaign  20 July,
        2021.
        
    \item “Who ya gonna call?”: Cybersecurity for the Spectre Era.
    California State University Northridge Virtual Research Presentations: Computer Science and Cyber Security.
    22 March, 2021.
    
\end{bibenum}
\end{document}