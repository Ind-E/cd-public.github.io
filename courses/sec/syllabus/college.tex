\subsection*{College Policies}

The following material is adapted from ``Information for Syllabus'' recommended language on syllabus prepartion provided to insturctors in the College of Arts \& Sciences.  The following sections represent the views of the instructors employer, rather than the instructor themselves, and have been lightly editted in some cases for clarity and sensativity.

\subsubsection*{Time Commitment}

Willamette's Credit Hour Policy holds that for every hour of class time there is an expectation of 2-3 hours’ work outside of class.  Thus, for this class you should anticipate spending 6-9 hours outside of class engaged in course-related activities. Examples include reading course materials, preparing for discussion, preparing and writing papers and exams.

\subsubsection*{Academic Integrity}

Students of Willamette University are members of a community that values excellence and integrity in every aspect of life. As such, we expect all community members to live up to the highest standards of personal, ethical, and moral conduct. Students are expected not to engage in any type of academic or intellectually dishonest practice and encouraged to display honesty, trust, fairness, respect, and responsibility in all they do. Plagiarism and cheating are especially offensive to the integrity of courses in which they occur and against the College community as a whole. These acts involve intellectual dishonesty, deception, and fraud, which inhibit the honest exchange of ideas. Plagiarism and cheating may be grounds for failure in the course and/or dismissal from the College. \url{http://willamette.edu/cla/catalog/policies/plagiarism-cheating.php}

\subsubsection*{Commitment to Positive Sexual Ethics}

Willamette is a community committed to fostering safe, productive learning environments, and we value ethical sexual behaviors and standards. Title IX and our school policy prohibit discrimination on the basis of sex, which regards sexual misconduct — including discrimination, harassment, domestic and dating violence, sexual assault, and stalking. We understand that sexual violence can undermine students’ academic success, and we encourage affected students to talk to someone about their experiences and get the support they need. 

\begin{quoting}\textbf{Please be aware that as a mandatory reporter I am required to report any instances you disclose to Willamette's Title IX Coordinator.}\end{quoting}

If you would rather share information with a confidential employee who does not have this responsibility, please contact our confidential advocate at confidential-advocate@willamette.edu. Confidential support also can be found with SARAs and at the GRAC (503-851-4245); and at WUTalk - a 24-hour telephone crisis counseling support line (503-375-5353). If you are in immediate danger, you may reach campus safety at 503-370-6911.

\subsubsection*{DACA/Undocumented Student Advocate}

Willamette is committed to supporting our DACA/Undocumented students in a variety of ways. This year, Tori Ruiz is the contact person for all DACA/undocumented students can provide those students with a number of external and internal resources that are available. Her contact information is email:~\href{mailto:truiz@willamette.edu}{truiz@willamette.edu}, Office: 3rd Floor UC, Phone: 503-370-6447.

\subsubsection*{Diversity and Disability Statement}

Willamette University values diversity and inclusion; we are committed to a climate of mutual respect and full participation. My goal is to create a learning environment that is usable, equitable, inclusive and welcoming. If there are aspects of the instruction or design of this course that result in barriers to your inclusion or accurate assessment or achievement, please notify me as soon as possible. Students with disabilities are also encouraged to contact the Accessible Education Services office in Smullin 155 at 503-370-6737 or Accessible-info@willamette.edu to discuss a range of options to removing barriers in the course, including accommodations.

If you are disabled person or person with a disability and have preference for indentity first or person first language, I would be grateful to be informed of your preference to best affirm you.

\subsubsection*{Religious Practice}

Willamette University recognizes the value of religious practice and strives to accommodate students’ commitment to their religious traditions whenever possible. Please let me know within the first two weeks of the semester if a conflict between holy days or other religious practice and full participation in the course is anticipated. I will do my best to work with you to determine a reasonable accommodation.

\textit{As an instructor, I will exercise my discretion to offer accomodations for conflicts after the first two weeks of the semester. You may always reach out to me, including retroactively, though the quality of the accomodation I am able to offer may improve given advanced warning!}

\subsubsection*{SOAR Center Offerings: Food, Clothing, and School Materials}

The Students Organizing for Access to Resources (SOAR) Center strives to create equitable access to food, professional clothing, commencement regalia, and scholarly resources for WU and Willamette Academy students. The SOAR Center is located on the Putnam University Center's third floor (in the former Women's Resource Center and across from the Harrison Conference Room). The space houses the Bearcat Pantry, Clothing Share, and First-Generation Book Drive and is maintained by committed students and staff and faculty advisers.

\subsubsection*{Trans Inclusion and Gender Justice}

I am always appreciative of the opportunity to address you by your affirmed name, pronouns, and any other gender markers. Please advise me of this at any point in the semester so that I may may best respect you at all times.

If I ever misgender you in any way, I would greatly appreciate that you let me know, in whatever manner makes you comfortable, so that I can correct that error and endeavour to repair any harm. 

\subsubsection*{Mental Health}

As a student, you may experience a range of challenges that can interfere with learning, such as strained
relationships, increased anxiety, substance use, feeling down, difficulty concentrating and/or lack of
motivation. These mental health concerns or stressful events may diminish your academic performance and/or
reduce your ability to participate in daily activities. Willamette services are available and treatment does work.
If you think you need help, please contact Bishop Health as soon as possible at
\url{http://willamette.edu/offices/counseling/}. Crisis counseling is available 24/7 at WUTalk: 503-375-5353 and
Campus Safety is available at 503-370-6911. Emergency resources are also available from the Psychiatric
Crisis Center at 503-585-4949 and the National Suicide Prevention Lifeline at 1-800-273-8255.


\subsubsection*{Intellectual Property \& Privacy}

Willamette's Credit Hour Policy holds that for every hour of class time there is an expectation of 2-3 hours’ work outside of class.  Thus, for this class you should anticipate spending 6-9 hours outside of class engaged in course-related activities. Examples include reading course materials, preparing for discussion, preparing and writing papers and exams.

Class materials and discussions including recorded lectures are for the sole purpose of educating the students enrolled in the course.  The release of such information (including but not limited to directly sharing, screen capturing, or recording content) is strictly prohibited, unless the instructor states otherwise. Doing so without the permission of the instructor will be considered an Honor Code violation and may also be a violation of other state and federal laws, such as the Copyright Act.

\textit{All of my course materials are open source. I will rely on some materials from our instructors, but believe they are all open source as well.}