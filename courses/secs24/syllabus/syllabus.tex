% Don't touch this %%%%%%%%%%%%%%%%%%%%%%%%%%%%%%%%%%%%%%%%%%%
\documentclass[11pt]{article}
\usepackage{fullpage}
\usepackage[left=1in,top=1in,right=1in,bottom=1in,headheight=3ex,headsep=3ex]{geometry}
\usepackage{graphicx}
\usepackage{float}
\usepackage{quoting}

\setlength{\parindent}{0em}
\setlength{\parskip}{1em}

\newcommand{\blankline}{\quad\pagebreak[2]}
%%%%%%%%%%%%%%%%%%%%%%%%%%%%%%%%%%%%%%%%%%%%%%%%%%%%%%%%%%%%%%

% Modify Course title, instructor name, semester here %%%%%%%%

\title{DATA-599: Cybersecurity}
\author{Calvin Deutschbein (they)}
\date{Willamette University, Spring 2023}

%%%%%%%%%%%%%%%%%%%%%%%%%%%%%%%%%%%%%%%%%%%%%%%%%%%%%%%%%%%%%%

% Don't touch this %%%%%%%%%%%%%%%%%%%%%%%%%%%%%%%%%%%%%%%%%%%
\usepackage[sc]{mathpazo}
\linespread{1.05} % Palatino needs more leading (space between lines)
\usepackage[T1]{fontenc}
\usepackage[mmddyyyy]{datetime}% http://ctan.org/pkg/datetime
\usepackage{advdate}% http://ctan.org/pkg/advdate
\newdateformat{syldate}{\twodigit{\THEMONTH}/\twodigit{\THEDAY}}
\newsavebox{\MONDAY}\savebox{\MONDAY}{Mon}% Mon
\newcommand{\week}[1]{%
%  \cleardate{mydate}% Clear date
% \newdate{mydate}{\the\day}{\the\month}{\the\year}% Store date
  \paragraph*{\kern-2ex\quad #1, \syldate{\today} - \AdvanceDate[4]\syldate{\today}:}% Set heading  \quad #1
%  \setbox1=\hbox{\shortdayofweekname{\getdateday{mydate}}{\getdatemonth{mydate}}{\getdateyear{mydate}}}%
  \ifdim\wd1=\wd\MONDAY
    \AdvanceDate[7]
  \else
    \AdvanceDate[7]
  \fi%
}
\usepackage{setspace}
\usepackage{multicol}
%\usepackage{indentfirst}
\usepackage{fancyhdr,lastpage}
\usepackage{url}
\pagestyle{fancy}
\usepackage{hyperref}
\usepackage{lastpage}
\usepackage{amsmath}
\usepackage{layout}

\lhead{}
\chead{}
%%%%%%%%%%%%%%%%%%%%%%%%%%%%%%%%%%%%%%%%%%%%%%%%%%%%%%%%%%%%%%

% Modify header here %%%%%%%%%%%%%%%%%%%%%%%%%%%%%%%%%%%%%%%%%
\rhead{\footnotesize Intro}

%%%%%%%%%%%%%%%%%%%%%%%%%%%%%%%%%%%%%%%%%%%%%%%%%%%%%%%%%%%%%%
% Don't touch this %%%%%%%%%%%%%%%%%%%%%%%%%%%%%%%%%%%%%%%%%%%
\lfoot{}
\cfoot{\small \thepage/\pageref*{LastPage}}
\rfoot{}

\usepackage{array, xcolor}
\usepackage{color,hyperref}
\definecolor{clemsonorange}{HTML}{EA6A20}
\hypersetup{colorlinks,breaklinks,linkcolor=clemsonorange,urlcolor=clemsonorange,anchorcolor=clemsonorange,citecolor=black}

\begin{document}

\maketitle

\blankline

\begin{tabular*}{.93\textwidth}{@{\extracolsep{\fill}}lr}

%%%%%%%%%%%%%%%%%%%%%%%%%%%%%%%%%%%%%%%%%%%%%%%%%%%%%%%%%%%%%%

% Modify information %%%%%%%%%%%%%%%%%%%%%%%%%%%%%%%%%%%%%%%%%
E-mail: \texttt{ckdeutschbein@willamette.edu} & Web: \href{https://cd-public.github.io/courses/intro/151s22.html}{\tt\bf cd-public.github.io/}  \\

 Office Hours: By Appt.  &  Lecture W 6:00-10:00 PM \\

 Office: Zoom, Ford 206 & Lecture Hall: Portland Center\\
 & \\
&  \\
\hline
\end{tabular*}

\vspace{5 mm}

% First Section %%%%%%%%%%%%%%%%%%%%%%%%%%%%%%%%%%%%%%%%%%%%

\section*{Course Description}

Cybersecurity can be understood as a mindset or approach rather than a subfield of computer science, such as secure mobile computing, network and operating system security, secure data bases, and secure cryptography algorithms. This course prepares a general audience to incorporate security concepts and ethics into their daily lives and offers some basic familiarity with writing security oriented code. 

% Second Section %%%%%%%%%%%%%%%%%%%%%%%%%%%%%%%%%%%%%%%%%%%

\section*{Required Materials}

Lecture materials will be available on the course webpage under \href{https://cd-public.github.io/courses/sec/sched.html}{schedule}.


% Third Section %%%%%%%%%%%%%%%%%%%%%%%%%%%%%%%%%%%%%%%%%%%

\section*{Prerequisites}

This class is meant to open to all students.

\section*{Accessability}

I will make every effort to ensure all coursework and materials are accessible to all students, including working with on-campus specialists. However, there is always room for improvement. I always appreciate hearing from students about how I can make the course more accessible, so please reach out if there is something I can be doing better!

% Fourth Section %%%%%%%%%%%%%%%%%%%%%%%%%%%%%%%%%%%%%%%%%%%

\section*{Course Objectives}
This course will teach you techniques for reasoning about information and computing and controlled accesses to these resources. As a survey course of the broad discipline of computer security, it will focus on different abstraction levels, from cryptographic code at a low level to the cultural and economic implications of secure and insecure data access at a high level.
\begin{itemize}
\item You will practice styles of thinking used by security researchers to contextualize their work in the broader context of computing and society.
\item You will gain experience working with common coding practices for security, especially in the context of network and internet security.
\item You will learn some historical efforts to attack and defend various computing systems, and discuss the implications of the state of computer security as a discipline and as deployed in practice..
\item You will learn some theoretical background in formulating notions of security (the ``logical foundations'' of computer security).
\item You will be exposed to state-of-the-art security research specific to hardware designs, including computer processors, as an example of ongoing research efforts.
\end{itemize}
This course will equip you apply notions of computer security to your other coursework, within computer science as well as within the college, and empower you to be a responsible computer scientist and member of an increasingly computer reliant society.

% Fifth Section %%%%%%%%%%%%%%%%%%%%%%%%%%%%%%%%%%%%%%%%%%%

\section*{Course Structure}

The course will be composed of lectures, labs, homeworks, midterms, and a final project.

\subsection*{Lecture Structure}

Lectures are scheduled for Wednesdays at 6:-0 PM in the Portland Center. The schedule
of lectures will be available on the course webpage under \href{https://cd-public.github.io/courses/sec/sched.html}{schedule}.

\bigskip
\noindent Lectures will be primarily on the white board, with some coding. While I will make every effort to follow best practices for accessible teaching, I will make mistakes! Please, if you find some material is inaccessible for any reason
do not hesitate to reach out.

\subsubsection*{Midterm Structure}

Feedback will be provided on midterm exams.  Midterm exams will be considered when determining grades for this course.

\bigskip
\noindent There will be two written midterm final exams, intended to be completed in small groups
without access to notes or documentation. The midterms are intended to achieve a learning
focus of reasoning about security in isolation from course materials,
as well provide me as an instructor with greater insight into how effective course
instruction has been.

\noindent Feedback scores will constitute the minimum grade on an assignment, but the instructor
may exercise discretion at any time to award a higher grade. For example, a submitted homework
may not use some important algorithmic technique as submitted, but if the student showed familiarity
with this technique on an earlier assignment or exam, the absence of that technique in a specific
case need not be counted against a student in grading, but only noted in feedback. This corresponds to the high level notion of feedback corresponding to how well an assignment reached the intended learning goals, while the overall course grade is meant to indicate that a student is prepared to succeed in latter coursework. Under this model, the final project will offer an opportunity to show familiarity with all content in the course, so a strong final project can ensure a high course grade
for any student, regardless of prior scores on midterms and homeworks.

\subsection*{College Policies}

The following material is adapted from ``Information for Syllabus'' recommended language on syllabus prepartion provided to insturctors in the College of Arts \& Sciences.  The following sections represent the views of the instructors employer, rather than the instructor themselves, and have been lightly editted in some cases for clarity and sensativity.

\subsubsection*{Time Commitment}

Willamette's Credit Hour Policy holds that for every hour of class time there is an expectation of 2-3 hours’ work outside of class.  Thus, for this class you should anticipate spending 6-9 hours outside of class engaged in course-related activities. Examples include reading course materials, preparing for discussion, preparing and writing papers and exams.

\subsubsection*{Academic Integrity}

Students of Willamette University are members of a community that values excellence and integrity in every aspect of life. As such, we expect all community members to live up to the highest standards of personal, ethical, and moral conduct. Students are expected not to engage in any type of academic or intellectually dishonest practice and encouraged to display honesty, trust, fairness, respect, and responsibility in all they do. Plagiarism and cheating are especially offensive to the integrity of courses in which they occur and against the College community as a whole. These acts involve intellectual dishonesty, deception, and fraud, which inhibit the honest exchange of ideas. Plagiarism and cheating may be grounds for failure in the course and/or dismissal from the College. \url{http://willamette.edu/cla/catalog/policies/plagiarism-cheating.php}

\subsubsection*{Commitment to Positive Sexual Ethics}

Willamette is a community committed to fostering safe, productive learning environments, and we value ethical sexual behaviors and standards. Title IX and our school policy prohibit discrimination on the basis of sex, which regards sexual misconduct — including discrimination, harassment, domestic and dating violence, sexual assault, and stalking. We understand that sexual violence can undermine students’ academic success, and we encourage affected students to talk to someone about their experiences and get the support they need. 

\begin{quoting}\textbf{Please be aware that as a mandatory reporter I am required to report any instances you disclose to Willamette's Title IX Coordinator.}\end{quoting}

If you would rather share information with a confidential employee who does not have this responsibility, please contact our confidential advocate at confidential-advocate@willamette.edu. Confidential support also can be found with SARAs and at the GRAC (503-851-4245); and at WUTalk - a 24-hour telephone crisis counseling support line (503-375-5353). If you are in immediate danger, you may reach campus safety at 503-370-6911.

\subsubsection*{DACA/Undocumented Student Advocate}

Willamette is committed to supporting our DACA/Undocumented students in a variety of ways. This year, Tori Ruiz is the contact person for all DACA/undocumented students can provide those students with a number of external and internal resources that are available. Her contact information is email:~\href{mailto:truiz@willamette.edu}{truiz@willamette.edu}, Office: 3rd Floor UC, Phone: 503-370-6447.

\subsubsection*{Diversity and Disability Statement}

Willamette University values diversity and inclusion; we are committed to a climate of mutual respect and full participation. My goal is to create a learning environment that is usable, equitable, inclusive and welcoming. If there are aspects of the instruction or design of this course that result in barriers to your inclusion or accurate assessment or achievement, please notify me as soon as possible. Students with disabilities are also encouraged to contact the Accessible Education Services office in Smullin 155 at 503-370-6737 or Accessible-info@willamette.edu to discuss a range of options to removing barriers in the course, including accommodations.

If you are disabled person or person with a disability and have preference for indentity first or person first language, I would be grateful to be informed of your preference to best affirm you.

\subsubsection*{Religious Practice}

Willamette University recognizes the value of religious practice and strives to accommodate students’ commitment to their religious traditions whenever possible. Please let me know within the first two weeks of the semester if a conflict between holy days or other religious practice and full participation in the course is anticipated. I will do my best to work with you to determine a reasonable accommodation.

\textit{As an instructor, I will exercise my discretion to offer accomodations for conflicts after the first two weeks of the semester. You may always reach out to me, including retroactively, though the quality of the accomodation I am able to offer may improve given advanced warning!}

\subsubsection*{SOAR Center Offerings: Food, Clothing, and School Materials}

The Students Organizing for Access to Resources (SOAR) Center strives to create equitable access to food, professional clothing, commencement regalia, and scholarly resources for WU and Willamette Academy students. The SOAR Center is located on the Putnam University Center's third floor (in the former Women's Resource Center and across from the Harrison Conference Room). The space houses the Bearcat Pantry, Clothing Share, and First-Generation Book Drive and is maintained by committed students and staff and faculty advisers.

\subsubsection*{Trans Inclusion and Gender Justice}

I am always appreciative of the opportunity to address you by your affirmed name, pronouns, and any other gender markers. Please advise me of this at any point in the semester so that I may may best respect you at all times.

If I ever misgender you in any way, I would greatly appreciate that you let me know, in whatever manner makes you comfortable, so that I can correct that error and endeavour to repair any harm. 

\subsubsection*{Mental Health}

As a student, you may experience a range of challenges that can interfere with learning, such as strained
relationships, increased anxiety, substance use, feeling down, difficulty concentrating and/or lack of
motivation. These mental health concerns or stressful events may diminish your academic performance and/or
reduce your ability to participate in daily activities. Willamette services are available and treatment does work.
If you think you need help, please contact Bishop Health as soon as possible at
\url{http://willamette.edu/offices/counseling/}. Crisis counseling is available 24/7 at WUTalk: 503-375-5353 and
Campus Safety is available at 503-370-6911. Emergency resources are also available from the Psychiatric
Crisis Center at 503-585-4949 and the National Suicide Prevention Lifeline at 1-800-273-8255.


\subsubsection*{Intellectual Property \& Privacy}

Willamette's Credit Hour Policy holds that for every hour of class time there is an expectation of 2-3 hours’ work outside of class.  Thus, for this class you should anticipate spending 6-9 hours outside of class engaged in course-related activities. Examples include reading course materials, preparing for discussion, preparing and writing papers and exams.

Class materials and discussions including recorded lectures are for the sole purpose of educating the students enrolled in the course.  The release of such information (including but not limited to directly sharing, screen capturing, or recording content) is strictly prohibited, unless the instructor states otherwise. Doing so without the permission of the instructor will be considered an Honor Code violation and may also be a violation of other state and federal laws, such as the Copyright Act.

\textit{All of my course materials are open source. I will rely on some materials from our instructors, but believe they are all open source as well.}

\end{document}

